\documentclass{article}


\usepackage{arxiv}

\usepackage[utf8]{inputenc} % allow utf-8 input
\usepackage[T1]{fontenc}    % use 8-bit T1 fonts
\usepackage{hyperref}       % hyperlinks
\usepackage{url}            % simple URL typesetting
\usepackage{booktabs}       % professional-quality tables
\usepackage{amsfonts}       % blackboard math symbols
\usepackage{nicefrac}       % compact symbols for 1/2, etc.
\usepackage{microtype}      % microtypography
\usepackage{graphicx}
\usepackage{natbib}
\usepackage{doi}

\usepackage{amsmath}
\usepackage{nicematrix}
\usepackage{nicematrix}
\usepackage{cleveref}


\NiceMatrixOptions{cell-space-top-limit=5pt,cell-space-bottom-limit=5pt,columns-width=20pt}

\def\R{\mathbb{R}}
\def\xt{\tilde{x}}



\title{Linear programming project}

%\date{September 9, 1985}	% Here you can change the date presented in the paper title
\date{} 					% Or removing it

\author{
  \hspace{1mm}Dmitry Beresnev \\
	MS-DS1, Innopolis University\\
	\texttt{d.beresnev@innopolis.university} \\
	\And{}
  \hspace{1mm}Vsevolod Klyushev \\
	MS-DS1, Innopolis University\\
	\texttt{v.klyushev@innopolis.university}
}

\renewcommand{\undertitle}{\textbf{Group 2} Report for Optimization F24 course}
\renewcommand{\headeright}{}

\begin{document}
\maketitle


\section{Introduction}
Initial problem is formulated as following:
\begin{equation}\label{eq:init}
  \begin{aligned}
                 & \min\limits_{x' \in \R^p} {\| Ax'-y' \|}_1 \\
    \text{s.t. } & 0 \leq x' \leq 1
  \end{aligned}
\end{equation}
where $A \in \R^{m \times p}$ with $m \geq p$ --- message  encoding matrix,
$y'$ --- received encoded (noisy) message,
$x'$ --- encoded initial message to be find.

\section{Notations}

 {
  \renewcommand{\arraystretch}{1.5}
  \renewcommand{\tabcolsep}{10pt}
  \begin{table}[h]
    \centering
    \begin{tabular}{cc}
      \toprule
      \textbf{Notation}                          & \textbf{Meaning}                                     \\
      \midrule
      $e_i$                                      & unit vector with 1 at index $i$ and all other zeroes \\
      $1_n$                                      & vector of $n$ ones                                   \\
      $I_n$                                      & identity matrix of size $n \times n$                 \\
      $0_n$                                      & vector of $n$ zeroes                                 \\
      $0_{m \times n}$                           & zero matrix of size $m \times n$                     \\
      $x_i$ $\left( \text{or } {(Ax)}_i \right)$ & $i$-th component of vector $x$    (or $Ax$)          \\

      \bottomrule
    \end{tabular}
  \end{table}
 }


\section{Q1: Linear problem formulation}
Initial problem (\cref{eq:init}) is not linear as cost function
$ {\| Ax'-y' \|}_1 = \sum_{i=1}^{m} |{(Ax')}_i-y'_i|$, is not linear. However, this objective function is \textbf{piecewise linear convex} function. Therefore, each element $|{(Ax')}_i-y'_i| = \max({(Ax')}_i-y'_i, y'_i -{(Ax')}_i)$ can be substituted with new variable $z'$ with the following additional
constraints: $z_i \geq {(Ax')}_i-y'_i$ and $z_i \geq y'_i-{(Ax')}_i$.

So the following problem is \textbf{linear} and is equivalent to the initial one:
\begin{equation}\label{eq:linear}
  \begin{aligned}
                 & \min\limits_{x' \in \R^p, z \in \R^m} \sum_{i=1}^{m} z_i \\
    \text{s.t. } & x' \geq 0                                                \\
                 & x' \leq 1                                                \\
                 & z_i \geq {(Ax')}_i-y'_i, \ i = 1 \dots m                 \\
                 & z_i \geq y'_i-{(Ax')}_i, \ i = 1 \dots m                 \\
  \end{aligned}
\end{equation}

\section{Q2: Linear problem in standard form}

For the easier and more evident deviation of standard form of \Cref{eq:linear},
linear problem will be firstly rewritten in geometric form, and only then --- in standard. The obtained linear optimization problem in standard form will be equivalent to initial problem (\cref{eq:init}).

\subsection{Geometric form}
The equivalent \textbf{geometric} form of \Cref{eq:linear} is
\begin{equation}\label{eq:geom}
  \begin{aligned}
                 & \min\limits_{z' \in \R^{p+m}} c^T z' \\
    \text{s.t. } &
    \begin{pNiceArray}{c:c}
      I_p  & 0_{p \times m}  \\
      \hdottedline
      -I_p  & 0_{p \times m}  \\
      \hdottedline
      -A & I_m \\
      \hdottedline
      A & I_m \\
      \CodeAfter
      \UnderBrace[yshift=1.5mm,color=blue]{last-1}{last-last}{A'}
    \end{pNiceArray}
    z' \geq
    \begin{pNiceMatrix}
      0_p  \\
      -1_p \\
      -y'  \\
      y'   \\
      \CodeAfter
      \UnderBrace[yshift=1.5mm,color=blue]{last-1}{last-last}{b'}
    \end{pNiceMatrix},
  \end{aligned}
  \vspace{20pt}
\end{equation}
where
$c = \sum_{i=p+1}^{p+m} e_i \in \R^{(p+m)}$,
$b' \in \R^{2p+2m}$ and
$A'\in \R^{(2p+2m) \times (p+m)}$.

The first $p$ components of $z'$ correspond to the components of $x'$, and the next $m$ components correspond to the components of $z$ from \Cref{eq:linear}. Rows and columns of $A'$ representation in \Cref{eq:geom} are separated in blocks for clarity: the vertical separation is for $x'$ and $z$ correspondingly, and the horizontal separations denote corresponding constraints from \Cref{eq:linear}.

\subsection{Standard form}

Note that $z' = {(x', z)}^T$ from \Cref{eq:geom} is already non-negative, because $x' \geq 0$ by problem definition and $z \geq 0 $ by construction\footnote{Intuitively, $z$ substitutes the absolute value, so is non-negative. Formally, from \Cref{eq:linear}, $z \geq t$ and $z \geq -t$ for some $t$. So if $t \geq 0$, then $z \geq t \geq 0$, and if $t \leq 0$ then $z \geq -t \geq 0$ }. Therefore, to convert \Cref{eq:geom} to standard form, only introduction of slack variables is needed to get rid of inequality sign. The equivalent \textbf{standard} form of \Cref{eq:geom} is
\begin{equation}\label{eq:std}
  \begin{aligned}
                 & \min\limits_{\xt \in \R^{2p+3m}} c^T \xt    \\
    \text{s.t. } &
    \begin{pNiceArray}{c:c:c}
      -I_p  & 0_{p \times m} & -S^{1,p}  \\
      \hdottedline
      -A & I_m & -S^{p+1,p+m} \\
      \hdottedline
      A & I_m & -S^{p+m+1,p+2m}\\
      \CodeAfter
      \UnderBrace[yshift=1.5mm,color=blue]{last-1}{last-last}{A'}
    \end{pNiceArray}
    z' =
    \begin{pNiceMatrix}
      -1_p \\
      -y'  \\
      y'   \\
      \CodeAfter
      \UnderBrace[yshift=1.5mm,color=blue]{last-1}{last-last}{b'}
    \end{pNiceMatrix}, \\
    \vspace{20pt}                                              \\
                 & \xt \geq 0,
  \end{aligned}
  \vspace{20pt}
\end{equation}
where
$c = \sum_{i=p+1}^{p+m} e_i \in \R^{(2p+3m)}$,
$b' \in \R^{p+2m}$,
$A'\in \R^{(p+2m) \times (2p+3m)}$, and
$S^{a,b}$ --- slack variable matrix of size $(b-a+1) \times (p+2m)$
with rows $S^{a,b}_i = e_{a+i-1}$, which represents necessary slack variables.

The first $p$ components of $\xt$ correspond to the components of $x'$, the next $m$ components correspond to the components of $z$ from \Cref{eq:linear} and the last $(p+2m)$ components correspond to slack variables $s$. Rows and columns of $A'$ representation in \Cref{eq:std} are again separated in blocks for clarity: the vertical separations are for $x'$, $z$ and $s$ correspondingly, and the horizontal separations are related to corresponding constraints from \Cref{eq:geom} (except first one, as non-negativity in standard from is separate constraint).


\end{document}
